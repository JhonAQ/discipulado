{\color{gray}\hrule}
\begin{center}
\section{Selección del pasaje}
\textbf{En esta sección se comparan las distintas versiones del pasaje, y se limita el texto.}
\end{center}
{\color{gray}\hrule}

\begin{multicols}{2}

\subsection{Delimitando el texto}

  Para iniciar, se seleccionó el texto que habla sobre el buen Pastor,
  para esto pudimos ver que hay muchas referencias indirectas o parciales en la biblia sobre
  el buen Pastor, pero el pasaje que lo señala de manera directa se encuentra en \cite[Juan 10:11-18]{BibliaNVI}.

  Después de delimitar el pasaje de forma general, se procede a hacer las comparaciones entre las distintas versiones
  que existen, para este caso se usaron la \cite[Biblia NVI]{BibliaNVI}, la \cite[Biblia RV]{BibliaRV}, la \cite[Biblia RVC]{BibliaRVC}, y la \cite[Biblia TLA]{BibliaTLA}.
  De donde se puede ver que no existen diferencias en la traducción del griego al español, es decir, \textbf{todas las traducciones coinciden en el mensaje traducido},
  por lo que podemos perder cuidado en traducir el pasaje completo del griego.

  Para hacer la perícopa, se seleccionó solamente aquellos versículos que abordan directamente el tema del buen Pastor, ya que no queremos
  que nuestro análisis hermeneutico se exceda demasiado, en este sentido, los versículos que fueron seleccionados son los siguientes: 
  \begin{quotation}
    11 Yo soy el buen pastor; el buen pastor su vida da por las ovejas. 

    12 Mas el asalariado, y que no es el pastor, de quien no son propias las ovejas, ve venir al lobo y deja las ovejas y huye, y el lobo arrebata las ovejas y las dispersa. 

    13 Así que el asalariado huye, porque es asalariado, y no le importan las ovejas. 

    14 Yo soy el buen pastor; y conozco mis ovejas, y las mías me conocen, 

    15 así como el Padre me conoce, y yo conozco al Padre; y pongo mi vida por las ovejas. 

    16 También tengo otras ovejas que no son de este redil; aquellas también debo traer, y oirán mi voz; y habrá un rebaño, y un pastor. 

    17 Por eso me ama el Padre, porque yo pongo mi vida, para volverla a tomar. 

    18 Nadie me la quita, sino que yo de mí mismo la pongo. Tengo poder para ponerla, y tengo poder para volverla a tomar. Este mandamiento recibí de mi Padre.
  \end{quotation}

\end{multicols}
