{\color{gray}\hrule}
\begin{center}
\section{Análisis del texto}
\textbf{Abordaremos el significado del "Buen Pastor"}
\end{center}
{\color{gray}\hrule}

\begin{multicols}{2}


\subsection{Significado del Buen Pastor}

  En el pasaje, Jesús se identifica a sí mismo como el "buen pastor", y sus palabras son una declaración directa de su misión y de quién es Él. 
  En términos generales, el "buen pastor" se diferencia del "asalariado", quien no se preocupa por las ovejas, mientras que el buen pastor da su vida por ellas. 

  Este contraste establece una clara distinción entre los líderes religiosos que, según Jesús, no tienen un verdadero compromiso con el bienestar de las personas, 
  y Él mismo, que está dispuesto a sacrificarse por sus seguidores.

  El "buen pastor" simboliza la relación de amor y sacrificio entre Jesús y los creyentes. 
  Él no es simplemente un líder que guía; Él es el salvador que da su vida para la salvación de sus ovejas. 

  El término "dar su vida" es crucial en este pasaje, ya que apunta hacia la cruz y el sacrificio expiatorio de Jesús, 
  algo que se desarrollará completamente en su crucifixión.

\subsection{Jesús como el Buen Pastor}

  La declaración de Jesús en Juan 10:11-18 refuerza la idea de que Él es el Mesías prometido, y que su misión es salvar a las personas, 
  no solo de sus pecados, sino también de la amenaza espiritual y física representada por los "lobos", es decir, los enemigos espirituales, 
  los líderes corruptos y las fuerzas del mal. La analogía con las ovejas también subraya la dependencia de los creyentes en Jesús para su sustento y protección.

  Cuando Jesús dice, "Yo soy el buen pastor", está declarando que Él es el cumplimiento de las promesas de Dios de ser el pastor que guiaría
  y cuidaría a su pueblo. La figura del pastor en la tradición bíblica no solo simboliza un líder sabio y protector, sino 
  también un líder sacrificial que está dispuesto a ir más allá por el bien de sus seguidores.

\end{multicols}
