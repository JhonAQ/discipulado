{\color{gray}\hrule}
\begin{center}
\section{Contexto Histórico y Literario}
\textbf{Veremos el contexto en el que fue escrito, asi como la finalidad}
\end{center}
{\color{gray}\hrule}

\begin{multicols}{2}

  Este pasaje forma parte de una serie de enseñanzas de Jesús que se centran en su relación 
  con sus seguidores y su identidad divina. La referencia al "buen pastor", está profundamente
  conectada con el Antiguo Testamento, en especial con el Salmo 23, donde Dios es presentado como
  un pastor que guía a su pueblo. En el Nuevo Testamento, Jesús se presenta como el pastor que
  cuida de su rebaño.

  Además, el contexto de la festividad de la dedicación del Templo de Jesusalén también está presente,
  lo que implica un ambiente de debate sobre la identidad y autoridad de Jesús. Hay que recalcar también, que este capítulo
  presenta a los fariseos como los principales opositores de Jesús, lo que nos da una idea de la tensión
  que se empezaba a gestar entre estos grupos políticos. Los "ladrones" que se mencionan al inicio del pasaje
  se refieren a los fariseos, que se presentan como falsos pastores que no cuidan del rebaño.

\end{multicols}
