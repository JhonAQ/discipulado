{\color{gray}\hrule}
\begin{center}


\section{Periodo pre-cristiano}
\textbf{En esta sección se examinarán los orígenes de la hermenéutica y los problemas que enfrentaron sus primeros exponentes.}
\end{center}

{\color{gray}\hrule}
\begin{multicols}{2}

\subsection{Inicios de la hermenéutica}

La hermenéutica en el periodo pre-cristiano comienza con las primeras formas de interpretación de las escrituras, influenciada por varios factores. Entre estos, las barreras lingüísticas jugaron un papel crucial. La mezcla cultural llevó a que las escrituras se tradujeran también al arameo, y la necesidad de interpretar correctamente para quienes no dominaban el idioma exigía procedimientos rigurosos para garantizar una adecuada transmisión de las ideas.

\subsection{Problemas de la hermenéutica}

Con el tiempo, la interpretación de las escrituras se simplificó, dando lugar a lecturas superficiales y leyes en las sinagogas que carecían de valor adicional o aplicación práctica en la vida de los judíos.

Otro problema que enfrentaron los primeros grupos hermenéuticos, como el de Qumrán, fue la falta de consideración del contexto histórico en los pasajes bíblicos que interpretaban, lo que resultó en una interpretación literal y rígida.

Finalmente, un problema significativo entre los cristianos de Roma fue la influencia de la corriente helenística, que mezclaba las interpretaciones de la filosofía griega con las escrituras. Esta influencia griega llevó a buscar alegorías complejas en textos literales, lo que disminuyó la importancia del autor original y otorgó mayor relevancia al intérprete.

\end{multicols}
