{\color{gray}\hrule}
\begin{center}
\section{Período de la Reforma}
\textbf{Examinaremos un panorama general de la hermenéutica en los inicios y el desarrollo de la Reforma}
\end{center}
{\color{gray}\hrule}
\begin{multicols}{2}

Durante el período de la Reforma, se buscó transformar las bases de la interpretación de los textos, que hasta entonces residían en la tradición. Una de las máximas de la Reforma fue el principio de \textit{sola scriptura}, que otorgaba a la Biblia la autoridad completa sobre la interpretación (interpretación basada únicamente en las Escrituras).

Las formas en que se modificó la hermenéutica durante este período incluyen la traducción de la Biblia al alemán y el impulso al estudio bíblico. 

Sin embargo, la libre interpretación de las Escrituras dio lugar a conflictos sobre la autoridad interpretativa y a la proliferación de interpretaciones divergentes dentro de la iglesia.

\end{multicols}
