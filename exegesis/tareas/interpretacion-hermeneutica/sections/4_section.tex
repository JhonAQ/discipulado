{\color{gray}\hrule}
\begin{center}
\section{Período Medieval}
\textbf{Examinaremos un panorama general de los aspectos positivos y negativos de la época}
\end{center}
{\color{gray}\hrule}
\begin{multicols}{2}

Durante el período medieval, la tradición predominó entre las interpretaciones hermenéuticas, estableciéndose por primera vez un sistema oficial de interpretación textual conocido como la cuatripartita. Este sistema consistía en cuatro formas de interpretar un texto: Literal, moral, alegórica y análoga. De este modo, se podían abordar los diferentes aspectos de un mensaje complejo expresado en algún pasaje.

\subsection{La Etapa Oscura}

Sin embargo, se vivió una etapa conocida como la ``Etapa oscura'' del cristianismo, en la cual las interpretaciones se realizaban casi sin consultar la Biblia. Una frase representativa de este período es:

\begin{quote}
"Primero aprende lo que debes creer, luego encuéntralo en la Biblia."
\end{quote}
\begin{flushright}
- Hugo de San Víctor
\end{flushright}

La carga interpretativa recaía en la persona que buscaba transmitir un mensaje, utilizando la Biblia únicamente para justificar sus afirmaciones, como si se tratara de un reportero en busca de material para una portada llamativa.

\end{multicols}
