{\color{gray}\hrule}
\begin{center}
\section{Período Histórico-Crítico y Contemporáneo}
\textbf{En esta sección se abordarán los problemas que persisten actualmente}
\end{center}
{\color{gray}\hrule}
\begin{multicols}{2}

Se abordan conjuntamente los períodos histórico-crítico y contemporáneo, ya que los problemas en ambos casos son similares. En el período crítico surgieron dos corrientes de pensamiento que se alineaban con las ideas del neoliberalismo, popular entre la sociedad intelectual: el pietismo y el racionalismo. Estas corrientes de interpretación cometieron el error de restar autoridad divina a las Escrituras, atribuyéndolas a una creación puramente humana, y negaron la inspiración de la Biblia.

Por otro lado, la hermenéutica contemporánea enfrenta desafíos como el relativismo (la idea de que no existe una interpretación correcta, sino múltiples interpretaciones igualmente válidas) y las crecientes influencias de ideologías de género y progresismo que impactan en la iglesia.

\end{multicols}
