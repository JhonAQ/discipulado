{\color{gray}\hrule}
\begin{center}
\section{Una orden para el pueblo}
\textbf{Revisaremos inicialmente el pasaje bíblico.}
\end{center}
{\color{gray}\hrule}

\begin{multicols}{2}

El pasaje \cite{mainPasaje} está en el libro de Números, hacia la parte final del capítulo 15.

\begin{quote}
  37 Y Jehová habló a Moisés, diciendo:\\
  38 Habla a los hijos de Israel, y diles que se hagan franjas en los bordes de sus vestidos, por sus generaciones; y pongan en cada franja de los bordes un cordón de azul.\\
  39 Y os servirá de franja, para que cuando lo veáis os acordéis de todos los mandamientos de Jehová, para ponerlos por obra; y no miréis en pos de vuestro corazón y de vuestros ojos, en pos de los cuales os prostituyáis.\\
  40 Para que os acordéis, y hagáis todos mis mandamientos, y seáis santos a vuestro Dios.\\
  41 Yo Jehová vuestro Dios, que os saqué de la tierra de Egipto, para ser vuestro Dios. Yo Jehová vuestro Dios.
\end{quote}

\subsection{Análisis del Contexto Histórico y Cultural}

\subsubsection{Contexto Histórico}
Este pasaje se encuentra en el libro de Números, que forma parte de los cinco libros de Moisés, también conocidos como la Torá o Pentateuco. Estos libros se escribieron durante el período en el cual los israelitas estaban en su travesía por el desierto tras haber salido de Egipto, y contienen leyes y directrices para la vida de la comunidad bajo el pacto con Dios.

\subsubsection{Contexto Cultural}
En el texto se mencionan los flecos o franjas (en hebreo: \textit{tzitzit}), que eran un símbolo visible en las prendas de los israelitas, que al ser visibles, servirían de recordatorio constante de su relación de pacto con Dios.
La mención que se hace de un cordón azul (en hebreo: \textit{tekhelet}) es de forma significativa, ya que el azul era un color que se asociaba con la realeza y la santidad en la cultura hebrea, reflejando la presencia de Dios y la separación de Israel como pueblo santo.

\subsection{Análisis Literario}

\subsubsection{Género Literario}
Este pasaje se da a modo de ley ceremonial \footnote{En el pasaje se menciona que esto lo deberan hacer todas las generaciones posteriores}. Forma parte de una serie de mandatos divinos que buscan regular la vida religiosa y moral en Israel.

\subsubsection{Estructura del Texto}
El texto se presenta en forma de discurso divino dirigido a Moisés, quien a su vez debe decirles estas reglas al pueblo. Podemos ver un patrón de mandato - reacción - beneficio: Dios da un mandato (que sería usar los flecos), la reacción esperada (recordar los mandamientos), y el beneficio (mantener la santidad y obediencia).

\end{multicols}
