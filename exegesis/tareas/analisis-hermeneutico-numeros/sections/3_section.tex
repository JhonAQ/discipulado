{\color{gray}\hrule}
\begin{center}
\section{Estudio de Palabras y Terminología}
\textbf{Ahora pasaremos al análisis de palabras clave para discernir mejor el significado del texto.}
\end{center}
{\color{gray}\hrule}

\begin{multicols}{2}

  \begin{itemize}
    \item \textbf{Flecos (tzitzit):} Simbolizan el llamado constante a recordar y obedecer los mandamientos de Dios. Eran una característica física que recordaba a los israelitas su identidad como pueblo elegido.
    \item \textbf{Cordón azul (tekhelet):} El uso del azul tiene significados teológicos profundos, relacionado con los cielos y, por lo tanto, con la presencia divina. El azul también simbolizaba pureza y fidelidad.
    \item \textbf{Prostituirse:} Aquí se usa en sentido figurado, refiriéndose a desviarse en la obediencia a Dios siguiendo deseos propios o influencias ajenas.
  \end{itemize}

\end{multicols}
