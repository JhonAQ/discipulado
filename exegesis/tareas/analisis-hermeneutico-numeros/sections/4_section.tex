\newpage
{\color{gray}\hrule}
\begin{center}
\section{Exégesis del pasaje}
\end{center}
{\color{gray}\hrule}

\begin{multicols}{2}

\subsection{Exégesis del pasaje}

\subsubsection{Significado Literal}

Dios instruye a los israelitas a usar flecos en sus vestiduras, con un cordón azul, como un recordatorio constante de Su ley y de su especial relación con Él. La finalidad de esto es que no sigan sus propios deseos o los de las naciones vecinas, lo cual es considerado como una forma de prostitución espiritual.

\subsubsection{Temas y Mensajes}

Los temas clave son la \textbf{obediencia} a los mandamientos de Dios, la santidad del pueblo de Israel y la \textbf{memoria} de su identidad como pueblo redimido por Dios. También nos dice la importancia la \textbf{fidelidad} a Dios en lugar con seguir los propios deseos o influencias externas.

\subsubsection{Relación con Otros Textos}

Este mandato es tambien similar a otras leyes ceremoniales que tienen como finalidad marcar a Israel y distinguirla de otras naciones paganas que existían.

\subsection{Aplicación Hermenéutica}

\subsubsection{Significado para la Audiencia Original}

Para los israelitas, este mandato servía como un recordatorio físico de su pacto con Dios y su llamado a la santidad y obediencia. Era una señal visible de su identidad como pueblo elegido.

\subsubsection{Aplicación Actual}

Aunque hoy en día no estamos obligados a usar flecos azules en nuestras ropas, el principio que está por debajo sigue siendo relevante: vivir con recordatorios visibles y conscientes de la relación con Dios y de la llamada a la obediencia y santidad.
Esto puede hacerce a través de la meditación diaria en la palabra, y las cosas que nos hacen recordar de Dios a nivel personal, asi como la oración constante para acercarnos cada dia a él.

\end{multicols}
