{\color{gray}\hrule}
\begin{center}
\section{Período Cristiano Primitivo}
\textbf{En esta sección revisaremos dos corrientes predominantes: el Alegorismo y el Literalismo}
\end{center}
{\color{gray}\hrule}
\begin{multicols}{2}

El período de la Iglesia Primitiva sentó las bases de la tradición eclesial que aún perdura en la Iglesia Católica. Los primeros "Padres de la Iglesia" realizaron revisiones, exposiciones y comentarios sobre numerosos pasajes bíblicos. A continuación, exploramos algunas de las corrientes y sus problemas:

\subsection{Corriente Alegórica}

La corriente alegórica surge de las interpretaciones helenísticas y, como tal, algunas interpretaciones llegaron a ser llevadas al extremo. Un ejemplo claro es el de Bernabé, quien interpretó el pasaje en el que se prohíbe a los judíos comer carne de águila, cuervo o cerdo. Según sus palabras, el autor quiso decir algo como: "No estés con una persona que es como un cerdo."

\subsection{Corriente Literal}

La corriente literal fue la más sensata y apegada a las Escrituras, prefiriendo, siempre que fuera posible, una interpretación literal sobre una alegórica. Sus mayores representantes fueron Clemente de Alejandría e Ignacio de Alejandría, quienes posteriormente abandonaron esta corriente para adoptar connotaciones alegóricas en sus interpretaciones.

\end{multicols}
